\documentclass{beamer}
%
% Choose how your presentation looks.
%
% For more themes, color themes and font themes, see:
% http://deic.uab.es/~iblanes/beamer_gallery/index_by_theme.html
%
\mode<presentation>
{
  \usetheme{Madrid}      % or try Darmstadt, Madrid, Warsaw, ...
  \usecolortheme{beaver} % or try albatross, beaver, crane, ...
  \usefonttheme{serif}  % or try serif, structurebold, ...
  \setbeamertemplate{navigation symbols}{}
  \setbeamertemplate{caption}[numbered]
} 

\usepackage[english]{babel}
\usepackage[utf8x]{inputenc}
\usepackage{xcolor}
\usepackage{listings}
\usepackage{algorithm2e}
\lstset
{
    language=[LaTeX]TeX,
    breaklines=true,
    basicstyle=\tt\scriptsize,
    %commentstyle=\color{green}
    keywordstyle=\color{blue},
    %stringstyle=\color{black}
    identifierstyle=\color{magenta},
}

\title[CS396]{Auto Scaling of Key Value stores}
\author{Raghukul Raman}
\institute{IIT Kanpur}
\date{\today}

\AtBeginSection[]
{
  \begin{frame}<beamer>
    \frametitle{Outline}
    \tableofcontents[currentsection,currentsubsection]
  \end{frame}
}

\begin{document}

\begin{frame}
  \titlepage
\end{frame}

% Uncomment these lines for an automatically generated outline.
\begin{frame}{Outline}
  \tableofcontents
\end{frame}

\section{Abstract}

\begin{frame}{Aim}
    \begin{itemize}
        \item Collect statistics for different queries on different scaling configurations
            and try to find possible bottlenecks in the system.
        \item Providing an auto-scaling solution for key
                value stores. 
    \end{itemize}
\end{frame}



\section{Introduction}
\begin{frame}{Key Value Stores}
    \begin{itemize}
        \item Data is organized in just 2 columns - a \textbf{key}, and a \textbf{value}.
        \item Actual data is value - can be object, keys are used to index these data objects.
        \item Satisfy \textbf{BASE} property.
            \begin{itemize}
                \item Basically Available
                \item Soft state
                \item Eventually consistent
            \end{itemize}
        \item \textbf{CAP} theorem
            \begin{itemize}
                \item Consistency
                \item Availability
                \item Partition tolerance
            \end{itemize}
        \item Fall into CP category.
        \item Mostly In-memory - extremy fast compared to traditional DBs.
    \end{itemize}
\end{frame}

\begin{frame}{Scalability}
    \begin{itemize}
        \item \textit{def:} Property of a system to handle a growing amount of work by adding resources
            to the system[Bondi, Andre - 2000].
        \item Two variants:
            \begin{itemize}
                \item \textbf{Vertical Scaling:} Increasing the resources in the
                        server which we are currently using, i.e increase the amount of memory, CPU etc.
                \item \textbf{Horizontal Scaling:} Increasing the number of servers (instances).
            \end{itemize}
        \item Benefits of horizontal scaling:
            \begin{itemize}
                \item make system fault tolerant.
                \item down time.
            \end{itemize}
        \item Several load balancing schemes are used in horizontal scaling.
    \end{itemize}
\end{frame}

\begin{frame}{Redis}
    \begin{itemize}
        \item developed by Salvatore Sanfilippo(\textit{antirez}).
        \item Open source, in-memory data structure store, used as a database, cache and message broker.
        \item Supports other data structures like: strings, hashes, lists, sets, sorted sets with range queries, bitmaps, hyperloglogs.
        \item Built in support for replication, on-disk persistance and automated partitioning with Redis cluster.
        \item Master-slave asynchronous replication.
        \item Transaction, expiration time (BigTable).
    \end{itemize}
\end{frame}

\begin{frame}{Redis Cluster}
    \begin{itemize}
        \item Collection of redis nodes
            \begin{itemize}
                \item Able to communicate among themselves.
                \item Able to respond to requests collectively.
            \end{itemize}
        \item Data is automatically sharded across multiple Redis nodes.
        \item Every redis node require 2 ports:
            \begin{itemize}
                \item Lower one is used to server clients.
                \item Other used for Cluster Bus (node-to-node communication).
            \end{itemize}
        \item CRC16 hashing system - $16384$ hash slots.
        \item Each cluster node is responsible for a subset of hash slots.
        \item Hash tags.
        \item After cluster meet, every node contain node - hash slot mapping.
    \end{itemize}
\end{frame}

\begin{frame}{Resharding}
\begin{itemize}
    \item Redis Cluster supports the ability to add and remove nodes while the cluster is running.
    \item Same basic mechanism can be used in order to rebalance the cluster, add or remove nodes.
    \item Moving hash slots through Cluster Bus.
    \item No down time - with the help of MOVED Error.
\end{itemize}
\end{frame}

\begin{frame}{Redirection}
\begin{itemize}
    \item Client knows nothing - can send request to any node.
    \item On recieving a request:
        \begin{itemize}
            \item return the value if serves that hash slot.
            \item MOVED error, if not served.
        \end{itemize}
    \item MOVED error response also contain IP:PORT of node serving that hash slot.
    \item Still is an overhead.
\end{itemize}
\end{frame}


\begin{frame}{Partioning}
\begin{itemize}
    \item The way how we shard data among different nodes of redis cluster.
\end{itemize}
\end{frame}

\section{Experiments}
\begin{frame}{intro}
hello2.5
\end{frame}


\section{Observations}
\begin{frame}{intro}
hello3
\end{frame}


\section{Further works}
\begin{frame}{intro}
hello4
\end{frame}





\end{document}
