

%=======================   Default Templete   ==================
\documentclass[a4paper]{article}
\usepackage{graphicx}

% file with some default definations
\input{structure.tex}
\usepackage{listings}
\lstset{language=Python, basicstyle=\normalsize\sffamily\linespread{0.8}, numbers=left, numberstyle=\small, stepnumber=1, numbersep=5pt}
\usepackage{fancyhdr}
\usepackage{pdfpages} 
\setlength{\parindent}{0pt}
\fancypagestyle{note_1}{\fancyfoot[R]{\textit{* These kind of auction are sealed bid auction.}}}
\pagestyle{fancy}
\fancyhf{}
\lhead{\textbf{\NAME\ (\ANDREWID)}}
\chead{\textbf{UGP Report}}
\rhead{\COURSE}


%==================Header details======================
\newcommand\NAME{Raghukul Raman}
\newcommand\ANDREWID{160538}
\newcommand\HWNUM{4}
\newcommand\COURSE{CS395}
%======================================================

% available formatted sections:
% - COMMAND LINE ENVIRONMENT: \begin{commandline} \end{commandline}
% - FILE CONTENTS ENVIRONMENT: \begin{file}[optional filename, defaults to "File"]
% - NUMBERED QUESTIONS ENVIRONMENT: \begin{question}[optional title]
% - WARNING TEXT ENVIRONMENT(can also be used for note): \begin{warn}[optional title, defaults to "Warning:"]
% - INFORMATION ENVIRONMENT(can be used to mention given details): \begin{info}[optional title, defaults to "Info:"]

%===============================================================
\begin{document}
\includepdf[page={1}]{first_page.pdf}
\section*{Abstract}
In the modern world where application response is a critical metric,
in memory databases are gaining wide popularity. In-memory databases primarily relies
on main memory for data storage. These databases are becoming a crucial part of many
applications due to the availability of less expensive RAM, and the demand for high response time.
Accessing data in memory eliminates seek time when querying the data,
which provides faster and more predictable performance than disk.
Some of the cases where in-memory databases perform exceptionally well are Queues,
sticky sessions, caching, real-time analysis of data, etc. \\

Generally, these in-memory databases are implemented as key-value stores;
some traditional ones are Redis, Memcached, Hazelcast, etc. Large scale caching
can be done by scaling the instances of these databases. One way is running a fixed number of instances,
while the other is to scale up/down based on the load that our sever gets.
This project aims at analyzing and providing an auto-scaling solution to this problem. \\

First, we try to collect statistics for different queries on different node
configurations and try to find possible bottlenecks in the cluster based system.
We analyze by scaling redis horizontally as well as vertically and try to explain the results.
For analyzing clusters we use an open source tool called Twemproxy
(a fast, light-weight proxy for Memcached and redis). We try to limit resources and
measure performance for these configurations. We also try to build a system which can
be used to interact with the cluster and efficiently shard the requests among
different instances. Second part of the project is to give a scaling algorithm based on machine learning
and some statistical predictions.




\pagebreak
\section*{Introduction}


\pagebreak
\section*{Design Details}



\pagebreak
\begin{thebibliography}{9}

\bibitem{treasury} 
Leonardo Bartolini and Carlo Cottarelli.
\textit{Designing Effective Auctions for Treasury Securities}. 
Current issues in \texttt{ECONOMICS} and \texttt{FINANCE}, \textit{Vol. 3 No. 9, July 97}

\bibitem{rbi} 
Ravi Shankar and Sanjoy Bose. 
\textit{Auctions of Government Securities in India –An Analysis.}
Reserve Bank of India Occasional Papers,\textit{ Vol. 29 No. 3, Winter 2008}

\bibitem{recent_adapt}
Kenneth D. Garbade and Jeffrey F. Ingber.
\textit{The Treasury Auction Process: Objectives, Structure, and Recent Adaptations.}
Current issues in \texttt{ECONOMICS} and \texttt{FINANCE}, \textit{Vol. 11 No. 2, Feb 06}

\bibitem{}
Paul R. Milgrom.
\textit{Auctions and Bidding: A Primer.}
Journal of Economic Perspective, \textit{Vol. 3 No. 3, Summer 89}

\bibitem{}
Paul R. Milgrom and Robert J. Weber.
\textit{A theory of auctions and competitive bidding - II.}
1981

\bibitem{}
Kevin Leyton-Brown, Paul R. Milgrom and Ilya Segal.
\textit{Economics and computer science of a radio spectrum reallocation.}
PNAS, \textit{Vol. 114 No. 28, 2017}


\bibitem{knuthwebsite} 
Paul R. Milgrom.
\textit{Business Activities.}
\\\texttt{http://www.milgrom.net/business-activities}
\end{thebibliography}
\end{document}
